\subsection*{Sposoby elektryzowania ciał}
\begin{itemize}
    \item nadmiar elektronów \(\implies\) ładunek ujemny
    \item niedobór elektronów \(\implies\) ładunek dodatni
\end{itemize}
\subsubsection*{Podział substancji}
\begin{itemize}
    \item metale (np. miedź)
        \begin{itemize}
            \item posiadają jony w~węzłach sieci krystalicznej
            \item na jeden taki jon przypada jeden elekrton swobodny (oderwany od atomu, elektrony swobodne tworzą gaz elektronowy, czyli gaz Fermiego --- \(10^5\)-\(10^6\frac{\m}{\s}\), \(T = 300\K\))
            \item jeśli przyłożymy do metalu naładowane ciało, to elektrony swobodne przemieszczą się w~stronę tego ciała --- indukcja elektrostatyczna
        \end{itemize}
    \item dielektryki (zbudowane z~atomów objętnych) --- jeżeli zbliżymy do nich ciało naładowane dodatnio, to w~obrębie każdego atomu dielektryka deformuje się chmura elektronowa i~powstają dipole elektryczne --- polaryzacja
        \begin{itemize}
            \item xero składa się z~wałka selenowego, który przyciąga toner, a~następnie wprasowuje go w~papier
        \end{itemize}
    \item izolatory --- prawie nie przewodzą prądu elektrycznego
\end{itemize}
\subsubsection{Rozkład ładunku na powierzchni}
Ładunek lubi gromadzić się na ostrzach, dlatego jako piorunochrony stosuje się zaostrzone druty (wynalazek Benjamina Franklina).
\begin{description}
    \item[\(\sigma\)] --- powierzchniowa gęstość ładunku
        \begin{equation*}
            \sigma = \frac{Q}{S} = \unit{\frac{\C}{\m^2}}
        \end{equation*}
    \item[\(R\)] --- promień krzywizny
        \begin{equation*}
            \sigma \sim \frac{1}{R}
        \end{equation*}
\end{description}
Urządzenia:
\begin{description}
    \item[generator van der Grafa] --- do generowania wysokich napięć:
        \begin{itemize}
            \item gumowa ,,nóżka'' --- ładuje się ujemnie
            \item metalowa czasza -- ładuje się dodatnio
        \end{itemize}
        Można robić eksperymenty:
        \begin{itemize}
            \item wytwarzanie burzy
        \end{itemize}
    \item[maszyna elektrostatyczna]
\end{description}
\subsubsection*{Zasada zachowania ładunku elektrycznego}
W~układzie izolowanym elektrycznie suma ładunków dodatnich i~ujemnych nie zmienia się w~czasie (ładunek jest zachowany). Oznacza to, że nie da się unicestwić ładunku.
\subsubsection{Prawo Coulomba}
Dotyczy siły oddziaływania między dwoma ładunkami kulistymi (punktowymi).
\begin{mathfigure*}
    \fillpoint*{-2, 0}[\(q_1\)][above];
    \drawvec (-2, 0) -- node[above]{\(-\vec{F_\C}\)} (-4, 0);
    \drawvec (3, 0) -- node[above]{\(\vec{F_\C}\)}(6, 0);
    \drawdist (-2, -1) -- node[below]{\(r\)} (3, -1);
    \draw (3, 0) circle[radius=1cm];
    \fillpoint*{3, 0}[\(q_2\)][above];
\end{mathfigure*}
\begin{gather*}
    F_\C = k \cdot \frac{q_1 \cdot q_2}{r^2}\\
    k = 9 \cdot 10^9\frac{\N\m^2}{\C^2}
    k = \frac{1}{4\pi\varepsilon_0}\\
\end{gather*}
\begin{description}
    \item[\(\varepsilon_0\)] --- przenikalność elektryczna próżni
        \begin{equation*}
            \varepsilon_0 = 8,85 \cdot 10^{-12}\frac{\Farad}{\m}
        \end{equation*}
    \item[\(\varepsilon_r\)] --- względna przenikalność elektryczna
        \begin{gather*}
            \varepsilon_r = \frac{\varepsilon}{\varepsilon_0}\\
            \varepsilon = \varepsilon_r \cdot \varepsilon_0
        \end{gather*}
        Dla wody wynosi \(\varepsilon_r = 81\)
\end{description}
\subsubsection*{Pole elektryczne}
\begin{description}
    \item[\(Q\)] --- ładunek wytwarzający pole
    \item[\(q\)] --- ładunek próbny, którym będziemy sprawdzać siły oddziaływania
    \item[\(\vec{E}\)] --- natężenie pola elektrycznego
        \begin{equation*}
            \vec{E}
                = \frac{\vec{F_\C}}{q}
                = \frac{\cancel{2}F_\C}{\cancel{2}q}
                = \unit{\frac{\N}{\C}}
                = \unit{\frac{\V}{\m}}
        \end{equation*}
\end{description}
Jak obrazować pole elektryczne?
\begin{itemize}
    \item metoda pióropuszy (miałeś, chamie, czapkę z~piór) --- skrawki dielektryka umieszczone w~polu centralnym będą się polaryzować i~układać wzdłuż linii pola, pokazując, w~którą stronę zwrócone jest natężenie
\end{itemize}
Dipol elektryczny:
\begin{description}
    \item[\(p\)] --- moment dipolowy
        \begin{equation*}
            p = Qr
        \end{equation*}
\end{description}
Pole jednorodne --- równoległe linie pola w~dielektryku.
