\subsection*{Powtórzenie z~termodynamiki}
\begin{equation*}
    v_\p{kw} = \sqrt{\frac{3RT}{M}}
\end{equation*}
\subsubsection*{Zadanie~1.}
\begin{statement}
    W~naczyniu znajduje się wodór pod ciśnieniem \(p = 1600\hPa\) i~temperaturze \(T = 100\celsius\). Ile cząsteczek tego gazu znajduje się w~jednostce objętości? Jaka jest średnia prędkość kwadratowa cząsteczek gazu?
\end{statement}
Dane:
\begin{gather*}
    p = 1600\hPa = 1,6 \cdot 10^5\Pa\\
    T = 100\celsius = 373\K\\
    M_{\p{H}_2} = 0,002\frac{\kg}{\mol}
\end{gather*}
Podstawiamy do wzoru
\begin{equation*}
    v_\p{kw.} = \sqrt{\frac{3 \cdot 8,31 \cdot 373}{0,002}} = 2156\frac{m}{s}
\end{equation*}
Liczbę moli możemy wziąć z~równania Clapeyrona:
\begin{gather*}
    pV = nRT\\
    n = \frac{pV}{RT}
    \frac{n}{V} = \frac{p}{RT}\\
    N_\p{A} \cdot \frac{n}{V} = \frac{p}{RT} \cdot N_\p{A} = \frac{1,6 \cdot 10^5}{8,31 \cdot 373} \cdot 6,022 \cdot 10^{23} = 3,1 \cdot 10^{25} \frac{\p{cz.}}{\m^3}
\end{gather*}
\begin{statement}
    Tę samą objętość gazu podgrzano izochorycznie do temperatury \(200\celsius\). Jakie jest teraz ciśnienie gazu? Natysuj wykres tej przemianu w~układach \(p\pars{V}\), \(V\pars{T}\), \(p\pars{T}\).
\end{statement}
\begin{gather*}
    V = \constfont{const}\\
    \frac{p_1}{T_1} = \frac{p_2}{T_2}\\
    p_2 = \frac{T_2}{T_1} \cdot p_1 = \frac{473\K}{373\K} \cdot 1,6 \cdot 10^5\Pa = 2,03 \cdot 10^5\Pa
\end{gather*}
\begin{mathfigure*}
    \drawaxes{0, 0}{2, 2}[\(V\)][\(p\)];
    \draw[] (1, 0.8) -- (1, 1.6);
\end{mathfigure*}
\begin{mathfigure*}
    \drawaxes{0, 0}{2, 2}[\(T\)][\(V\)];
    \draw (0.4, 1) -- (1.4, 1);
\end{mathfigure*}
\begin{mathfigure*}
    \drawaxes{0, 0}{2, 2}[\(T\)][\(p\)];
    \draw (0.5, 0.5) -- (2, 2);
    \draw[dashed] (0, 0) -- (0.5, 0.5);
\end{mathfigure*}
\subsubsection*{Trochę o~przemianach}
W~przemianie izochorycznej:
\begin{gather*}
    V = \constfont{const}\\
    \Delta U = \Delta Q = c_V \cdot n \cdot \Delta T
\end{gather*}
W~przemianie izobarycznej:
\begin{gather*}
    p = \constfont{const}\\
    \Delta U = \Delta Q_1 - \Delta W\\
    \Delta Q_1 = \Delta U + \Delta W\\
    \Delta Q_1 = n \cdot c_p \cdot \Delta T\\
\end{gather*}
Zatem \(c_p = c_V + R\).
\subsubsection*{Zadanie~2.}
\begin{statement}
    W~procesie izotermicznego rozprężania \(1,2\kg\) azotu (\(\p{N}_2\)) dostarczone zostało \(1200\J\) ciepła. Obliczyć, ile razy zmieniło się ciśnienie azotu, jeżeli jego temperatura początkowa była równa \(7\celsius\).
\end{statement}
\begin{gather*}
    m = 1,2\kg\\
    n = \frac{m}{M} = \frac{1200\g}{28\frac{\g}{\mol}} = 42,86\mol\\
    T_0 = 7\celsius = 280\K\\
    Q = W = nRT \ln \frac{V_k}{V_0}\\
    \frac{V_k}{V_0} = e^{\frac{Q}{nRT}} = e^{\frac{1,2 \cdot 10^3}{42,86 \cdot 8,31 \cdot 280}} = 1,012\\
    p_0 \cdot V_0 = p_k \cdot V_k\\
    \frac{V_k}{V_0} = \frac{p_k}{p_0}
\end{gather*}
\subsubsection*{Zadanie~3.}
\begin{statement}
    Na wykresie przedstawiono zamknięty cykl przemian gazu doskonałego o~cząsteczkach jednoatomowych. Oblicz w~każdej przemianie:
    \begin{enumerate}[label={\alph*)}]
        \item pracę siły zewnętrznej
        \item zmianę energii wewnętrznej gazu
        \item ciepło wymienione przez gaz z~otoczeniem
    \end{enumerate}
    \begin{mathfigure*}
        \drawaxes{0, 0}{5, 4}[\(V \unit{\dm^3}\)][\(p \unit{10^5\Pa}\)];
        \drawvec (3.6, 1) -- (1.2, 1);
        \drawvec (1.2, 1) -- (3.6, 3);
        \drawvec (3.6, 3) -- (3.6, 1);
        \fillpoint*{1.2, 1}[\(1\)][below left];
        \fillpoint*{3.6, 3}[\(2\)][above right];
        \fillpoint*{3.6, 1}[\(3\)][below right];
        \draw[dashed] (0, 3) node[left]{\(1,5\)} -- (3.6, 3);
        \draw[dashed] (0, 1) node[left]{\(0.5\)} -- (1.2, 1);
        \draw[dashed] (1.2, 0) node[below]{\(1.2\)} -- (1.2, 1);
        \draw[dashed] (3.6, 0) node[below]{\(3.6\)} -- (3.6, 1);
    \end{mathfigure*}
\end{statement}
Sprawność:
\begin{equation*}
    \eta = \frac{W}{Q_\p{pobr.}}
\end{equation*}
W~silniku idealnym
\begin{equation*}
    \eta = \frac{Q_\p{pobr} - Q_\p{odd.}}{Q_\p{pobr.}} = 1 - \frac{Q_\p{odd.}}{Q_\p{pobr.}}
\end{equation*}
Dla silnika Carnota
\begin{equation*}
    \eta = \frac{T_\p{g} - T_\p{ch}}{T_\p{g}} = 1 - \frac{T_\p{ch}}{T_\p{g}}
\end{equation*}
\begin{enumerate}[label={\alph*)}]
    \item liczymy odpowiednie pole pod / nad / obok wykresu
    \item
        \begin{itemize}
            \item
                \begin{equation*}
                    \Delta U = \frac{i}{2} \cdot N \cdot k_\p{B} \cdot \Delta T
                        = \frac{i}{2} \cdot n \cdot N_\p{A} \cdot k_\p{B} \cdot \Delta T = \frac{i}{2} \cdot nR \Delta T
                \end{equation*}
                Zapisujemy dwa razy równanie Clapeyrona:
                \begin{gather*}
                    p_1V_1 = nRT_1\\
                    p_2V_2 = nRT_2
                \end{gather*}
                Odejmujemy stronami:
                \begin{equation*}
                    p_2V_2 - p_1V_1 = nR\pars{T_2 - T_1} = nR \Delta T
                \end{equation*}
                Podstawiamy do wzoru na zmianę energii wewnętrznej:
                \begin{equation*}
                    \Delta U = \frac{i}{2} \cdot nR \Delta T
                        = \frac{3}{2} \cdot \pars{p_2V_2 - p_1V_1}
                \end{equation*}
        \end{itemize}
\end{enumerate}
