\subsection*{Ruch satelitów}
\subsubsection*{Zadanie}
\begin{statement}
    Asteroida o~masie \(m = 10^{10}\kg\) przelatuje koło Księżyca w~odległości \(60R_\p{Z}\) od środka Ziemi z~prędkością o~wartości \(v_0 = 10\frac{\km}{\s}\). Wyraź energię uderzenia w~Ziemię w~\(\Mega\ton\) trotylu (\(1\Mega\p{T} = 4 \cdot 10^{12}\J\)).
\end{statement}
Z~zasady zachowania energii:
\begin{gather*}
    E_\p{k k.} + E_\p{p k.} = E_{\p{k} 0} + E_{\p{p} 0}\\
    \frac{mv_\p{k}^2}{2} - G \cdot \frac{M_\p{Z}m}{R_\p{Z}} = \frac{mv_0^2}{2} - G \cdot \frac{M_\p{Z}m}{60R_\p{Z}}\\
    \frac{mv_\p{k}^2}{2} = \frac{mv_0^2}{2} - G \cdot \frac{M_\p{Z}m}{60R_\p{Z}} + G \cdot \frac{M_\p{Z}m}{R_\p{Z}}\\
    \begin{split}
        E_\p{k k.}
            &= \frac{mv_0^2}{2} + GM_\p{Z}m\pars{\frac{1}{R_\p{Z}} - \frac{1}{60R_\p{Z}}}
            = \frac{10^{10} \cdot 10^8}{2} + 6,67 \cdot 10^{-11} \cdot 6 \cdot 10^{24} \cdot 10^{10} \cdot 10^{-6}\pars{\frac{1}{6,371} - \frac{1}{60 \cdot 6,371}}\\
            &= \frac{1}{2} \cdot 10^{18} + 6,18 \cdot 10^{17}
            = 11,18 \cdot 10^{17}\J = 2,795 \cdot 10^{5}\Mega\p{T}
    \end{split}
\end{gather*}
\subsubsection*{Satelita geostacjonarny}
\begin{align*}
    F_\p{od} &= F_\p{g}\\
    \frac{mv^2}{r} &= \frac{GMm}{r^2}\\
    v^2 &= \frac{GM}{r}\\
    \frac{4\pi^2r^2}{T^2} &= \frac{GM}{r}\\
    \frac{r^3}{T^2} &= \frac{GM}{4\pi^2}\\
    r^3 &= \frac{GMT^2}{4\pi^2} = \frac{gR^2T^2}{4\pi^2}\\
    r &= \sqrt[3]{\frac{gR^2T^2}{4\pi^2}}\\
    T &= 24\h = 86400\s\\
    r &= \sqrt[3]{\frac{gR^2T^2}{4\pi^2}} = \sqrt[3]{\frac{10 \cdot 6371000^2 \cdot 86400^2}{4\pi^2}} = 42226393,32\m \approx 42226\km\\
    h &= r - R = 42226 - 6371 = 35855\km\\
    v &= \frac{2\pi r}{T} = \frac{2\pi \cdot 42226}{86400} \approx 3,07\frac{\km}{\s} = 11052\frac{\km}{\h}
\end{align*}
