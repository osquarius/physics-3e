\subsubsection*{Zadanie~7.2.1.}
\begin{mathfigure*}
    \draw[ultra thin, lightgray, step=0.25cm] (0, -1.25) grid (10.75, 2.5);
    \drawvec (0, 0) -- (10.75, 0) node[above left]{\scriptsize\(Q \unit{\kilo\J}\)};
    \drawvec (0, -1.25) -- (0, 2.5) node[left]{\scriptsize\(T \unit{\celsius}\)};
    \drawvmark{1.25, 0}[\scriptsize\(100\)][below];
    \drawvmark{2.5, 0}[\scriptsize\(200\)][below];
    \drawvmark{3.75, 0}[\scriptsize\(300\)][below];
    \drawvmark{5, 0}[\scriptsize\(400\)][below];
    \drawvmark{6.25, 0}[\scriptsize\(500\)][below];
    \drawvmark{7.5, 0}[\scriptsize\(600\)][below];
    \drawvmark{8.75, 0}[\scriptsize\(700\)][below];
    \drawvmark{10, 0}[\scriptsize\(800\)][below];
    \drawhmark{0, -1}[\scriptsize\(-40\)][left];
    \drawhmark{0, -0.5}[\scriptsize\(-20\)][left];
    \drawhmark{0, 0}[\scriptsize\(0\)][left];
    \drawhmark{0, 0.5}[\scriptsize\(20\)][left];
    \drawhmark{0, 1}[\scriptsize\(40\)][left];
    \drawhmark{0, 1.5}[\scriptsize\(60\)][left];
    \drawhmark{0, 2}[\scriptsize\(80\)][left];
    \draw[red, ultra thick] (0, -0.75) -- (0.75, 0) -- (4.875, 0) -- (10, 2.25) -- (10.5, 2.25);
\end{mathfigure*}
Z~wykresu odczytujemy, dla \(1\kg\) lodu, że aby przejść od \(-20\celsius\) do \(90\celsius\) potrzeba około \(740\kilo\J\) ciepła. Zatem dla \(6\kg\) potrzeba \(6 \cdot 740\kilo\J = 4440\kilo\J\) ciepła.
\subsubsection*{Zadanie~7.2.8.}
Dane:
\begin{gather*}
    m = 10\kg \qquad h = 1,5\m \qquad g = 9,81\frac{\m}{\s^2}\\
    m_\p{p} = 250\g = 0,25\kg \qquad c_\p{ż} = 452\frac{J}{\kg \cdot \K}\\
    \eta = 0,5 \qquad n = 50
\end{gather*}
Układamy bilans energetyczny, w~którym energia oddana to odpowiednia część energii kinetycznej młota (nabranej kosztem energii potencjalnej), a~energia pobrana to ciepło uzyskane przez pierścień:
\begin{align*}
    n \cdot E_\p{k} \cdot \eta &= Q\\
    n \cdot E_\p{p} \cdot \eta &= Q\\
    n \cdot mgh \cdot \eta &= m_\p{p} \cdot c_\p{ż} \cdot \Delta t\\
    \Delta t &= \frac{n \cdot mgh \cdot \eta}{m_\p{p} \cdot c_\p{ż}}
        = \frac{50 \cdot 10\kg \cdot 9,81\frac{\m}{\s^2} \cdot 1,5\m \cdot 0,5}{0,25\kg \cdot 452\frac{\J}{\kg \cdot \K}}
        \approx 32,56\K
\end{align*}
Temperatura pierścienia podniesie się o~około \(32,5\K\).
\subsubsection*{Zadanie~7.2.10.}
Dane:
\begin{gather*}
    m_\p{h} = 200\g = 0,2\kg \qquad c_\p{h} = c_\p{w} = 4200\frac{\J}{\kg \cdot \celsius} \qquad t_0 = 90\celsius
\end{gather*}
\begin{enumerate}[label={\alph*)}]
    \item
        \begin{gather*}
            m = 50\g = 0,05\kg \qquad t = 0\celsius
        \end{gather*}
        Układamy bilans cieplny, w~którym ciepło pobiera dolana woda, a~oddaje je herbata. Proces ten zachodzi aż do osiągnięcia wspólnej temperatury końcowej \(t_k\).
        \begin{align*}
            Q_\p{pobr.} &= Q_\p{odd.}\\
            m \cdot c_\p{w} \cdot \Delta t_1 &= m_\p{h} \cdot c_\p{h} \cdot \Delta t_2\\
            m \cdot c_\p{w} \cdot (t_k - t) &= m_\p{h} \cdot c_\p{h} \cdot (t_0 - t_k)\\
            t_k &= \frac{m_\p{h} \cdot c_\p{h} \cdot t_0 + m \cdot c_\p{w} \cdot t}{m \cdot c_\p{w} + m_\p{h} \cdot c_\p{h}}
                = \frac{0,2\kg \cdot 4200\frac{\J}{\kg \cdot \celsius} \cdot 90\celsius + 0,05\kg \cdot 4200\frac{\J}{\kg} \cdot 0\celsius}{0,05\g \cdot 4200\frac{\J}{\kg \cdot \celsius} + 0,2\kg \cdot 4200\frac{\J}{\kg \cdot \celsius}}
                = 72\celsius
        \end{align*}
        Herbata ochłodzi się do \(72\celsius\).
    \item
        \begin{gather*}
            m = 50\g = 0,05\kg \qquad t = 0\celsius\\
            c_\p{t} = 335000\frac{\J}{\kg}
        \end{gather*}
        Układamy bilans cieplny, w~którym ciepło pobiera lód, a~następnie powstała z~niego woda, a~oddaje je herbata. Proces ten zachodzi aż do osiągnięcia wspólnej temperatury końcowej \(t_k\).
        \begin{align*}
            Q_\p{pobr.} &= Q_\p{odd.}\\
            m \cdot c_\p{t} + m \cdot c_\p{w} \cdot \Delta t_1 &= m_\p{h} \cdot c_\p{h} \cdot \Delta t_2\\
            \begin{split}
                t_k &= \frac{m_\p{h} \cdot c_\p{h} \cdot t_0 + m \cdot c_\p{w} \cdot t - m \cdot c_\p{t}}{m \cdot c_\p{w} + m_\p{h} \cdot c_\p{h}}\\
                    &= \frac{0,2\kg \cdot 4200\frac{\J}{\kg \cdot \celsius} \cdot 90\celsius + 0,05\kg \cdot 4200\frac{\J}{\kg} \cdot 0\celsius - 0,05\kg \cdot 335000\frac{\J}{\kg}}{0,05\g \cdot 4200\frac{\J}{\kg \cdot \celsius} + 0,2\kg \cdot 4200\frac{\J}{\kg \cdot \celsius}} \approx 56,05\celsius
            \end{split}
        \end{align*}
        Herbata ochłodzi się do \(56,05\celsius\).
    \item Niższą temperaturę osiągniemy w~drugim sposobie, ponieważ przed oddawaniem ciepła na podgrzewanie wody herbata musi oddać jeszcze ciepło na roztopienie lodu bez zmiany temperatury.
\end{enumerate}
